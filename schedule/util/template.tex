\documentclass{article}
\usepackage{tikz}

\pdfpagewidth=8.5in		%-- Letter
\pdfpageheight=11in		%-- Letter
\textwidth=5.0in		%-- Letter
\textheight=9.5in		%-- Letter
 
\oddsidemargin -15mm
\evensidemargin -15mm

% solarized colors
\definecolor{yellow}{rgb}{0.70980392156863,0.53725490196078,0}
\definecolor{orange}{rgb}{0.796,0.294,0.0862}
\definecolor{red}{rgb}{0.86274509803922,0.19607843137255, 0.1843137254902}
\definecolor{magenta}{rgb}{0.82745098039216,0.21176470588235,0.50980392156863}
\definecolor{violet}{rgb}{0.42352941176471,0.44313725490196,0.76862745098039}
\definecolor{blue}{rgb}{0.14901960784314,0.54509803921569,0.82352941176471}
\definecolor{cyan}{rgb}{0.16470588235294,0.63137254901961,0.59607843137255}
\definecolor{green}{rgb}{0.52156862745098,0.6,0}

\begin{document}

% These set the width of a day and the height of an hour.
\newcommand*\daywidth{3.3cm}
\newcommand*\hourheight{3.5em}

% weekdays
\newcommand{\monday}{1}
\newcommand{\tuesday}{2}
\newcommand{\wednesday}{3}
\newcommand{\thursday}{4}
\newcommand{\friday}{5}


% The entry style will have two options:
% * the first option sets how many hours the entry will be (i.e. its height);
% * the second option sets how many overlapping entries there are (thus
%   determining the width).
% \tikzset{entry/.style 2 args={
%     draw,
%     rectangle,
%     anchor=north west,
%     line width=0.4pt,
%     inner sep=0.3333em,
%     text width={\daywidth/#2-0.6666em-0.4pt},
%     minimum height=#1*\hourheight,
%     align=center
% }}
\tikzset{course/.style 2 args={
    xshift=(0.5334em+0.8pt)/2,
    yshift=-0.1em,
    draw,
    line width=0.8pt,
    font=\sffamily,
    rectangle,
    rounded corners=.2ex,
    fill=gray!50,
    fill opacity=0.85,
    anchor=north west,
    inner sep=0.3333em,
    text width={\daywidth/#2-1.2em-1.6pt},
    minimum height=#1*\hourheight-0.3em,
    align=center
}}

\tikzset{office/.style 2 args={
    xshift=(0.5334em+0.8pt)/2,
    yshift=-0.1em,
    draw,
    line width=0.8pt,
    font=\sffamily,
    rectangle,
    rounded corners=.2ex,
    fill=gray!20,
    fill opacity=0.70,
    anchor=north west,
    inner sep=0.3333em,
    text width={\daywidth/#2-1.2em-1.6pt},
    minimum height=#1*\hourheight-0.3em,
    align=center
}}

\begin{center}

\begin{tabular}{ll}
  \begin{minipage}[l]{2in}
    \vtop{
      \textsf{
      {\huge !!!NAME!!!} \\
      {\large Schedule - Winter 2017} 
    }
    }
  \end{minipage} 
  &
    \begin{minipage}[l]{2in}
    \vtop{
    }
    }
  \end{minipage} 
\end{tabular}

\vspace{0.5in}


% Start the picture and set the x coordinate to correspond to days and the y
% coordinate to correspond to hours (y should point downwards).
\begin{tikzpicture}[y=-\hourheight,x=\daywidth]

    % First print a list of times.
    \foreach \time/\ustime in {8/8am,9/9am,10/10am,11/11am,12/12pm,13/1pm,14/2pm,15/3pm,16/4pm,17/5pm,18/6pm}
        \node[anchor=north east] at (1,\time-0.25) {\textsf{\ustime}};

    \foreach \time in {8.5,9.5,10.5,11.5,12.5,13.5,14.5,15.5,16.5,17.5}
        \node[anchor=north east] at (1, \time-0.25) {\textsf{\tiny{:30}}};


    \foreach \a in {8,9,10,11,12,13,14,15,16,17,18}
      \draw[gray, densely dashed] (1,\a) -- (6, \a);

    \foreach \a in {8.5,9.5,10.5,11.5,12.5,13.5,14.5,15.5,16.5,17.5}
      \draw[gray, dotted] (1,\a) -- (6, \a);

    % Draw some day dividers.
    \draw (1,6.5) -- (1,19);
    \draw (2,6.5) -- (2,19);
    \draw (3,6.5) -- (3,19);
    \draw (4,6.5) -- (4,19);
    \draw (5,6.5) -- (5,19);
    \draw (6,6.5) -- (6,19);

    % Day labels.
    \node[anchor=north] at (1.5,6.5) {\textsf{Monday}};
    \node[anchor=north] at (2.5,6.5) {\textsf{Tuesday}};
    \node[anchor=north] at (3.5,6.5) {\textsf{Wednesday}};
    \node[anchor=north] at (4.5,6.5) {\textsf{Thursday}};
    \node[anchor=north] at (5.5,6.5) {\textsf{Friday}};
   

    % Write the entries. Note that the x coordinate is 1 (for Monday) plus an
    % appropriate amount of shifting. The y coordinate is simply the starting
    % time.

    !!!SCHEDULE!!!
    
\end{tikzpicture}

\end{center}

\end{document}